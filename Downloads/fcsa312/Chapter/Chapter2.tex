\chapter{Хөгжүүлэлтэд ашиглах технологи, хэрэгслүүд}
\label{chapter2}

Энэхүү системийг хөгжүүлэхэд дараах технологиуд болон программ хангамжуудыг сонгосон болно.

\section{Үйлдлийн Системийн Сонголт}
Системийн хөгжүүлэлт болон байршуулалтад тохиромжтой үйлдлийн системийн хувьд \textbf{Linux} болон \textbf{macOS}-г сонгосон. Linux үйлдлийн систем нь нээлттэй эх үүсвэртэй бөгөөд сервер дээр ажиллахад тохиромжтой бол \textbf{macOS} нь хөгжүүлэгчийн орчинд илүү тохиромжтой байдаг.

\section{Өгөгдлийн Сан Удирдах Системийн Сонголт}
Өгөгдлийн сангийн хувьд \textbf{MySQL}-ийг сонгосон. Энэ нь өндөр гүйцэтгэлтэй, найдвартай бөгөөд олон хэрэглэгчийн өгөгдлийг нэгэн зэрэг боловсруулж чаддаг онцлогтой. Мөн шаардлагатай үед \textbf{PostgreSQL}-ийг ашиглах боломжтой.

\section{Программчлалын Хэлний Сонголт, Ашиглах Фреймворк}
Программчлалын хэлний хувьд:
\begin{itemize}
    \item \textbf{JavaScript}: Клиент болон сервер талын хөгжүүлэлтэд ашиглах бөгөөд \textbf{React} (урд талын хөгжүүлэлт) болон \textbf{Node.js} (арын хөгжүүлэлт)-тэй хослуулах.
    \item \textbf{Python}: Өгөгдөл боловсруулалт, машин сургалтын алгоритм боловсруулахад ашиглана.
\end{itemize}

Фреймворкууд:
\begin{itemize}
    \item \textbf{React}: Хэрэглэгчийн интерфэйсийг хөгжүүлэхэд ашиглах фреймворк. Энэхүү фреймворк нь хэрэглэгчийн туршлага сайжруулж, хурдан үзүүлэлтийг хангахад тусална.
    \item \textbf{Node.js}: Арын хөгжүүлэлтийг дэмжих бөгөөд өгөгдөл дамжуулах, сервер талын үйлчилгээг үр ашигтай гүйцэтгэхэд туслана.
    \item \textbf{Django}: Python хэл дээр суурилсан энэхүү фреймворк нь өгөгдөл боловсруулах, удирдах модульд ашиглагдах болно.
\end{itemize}

\section{Нэмэлт Программ Хангамж}
Системийн хөгжүүлэлт, чанарын хяналтыг сайжруулахад дараах нэмэлт программ хангамжуудыг ашиглана:
\begin{itemize}
    \item \textbf{Git}: Хувилбар удирдах систем; хөгжүүлэлтийн явцад кодын хувилбар болон хамтран ажиллах боломжийг хангана.
    \item \textbf{Docker}: Тохиргооны төвөг багатай, зөөврийн орчныг бий болгож хөгжүүлэлтийн хялбаршуулсан орчныг хангана.
    \item \textbf{VS Code}: Хөгжүүлэлтийн орчинг бүрдүүлэхэд тохиромжтой бөгөөд түгээмэл хэрэглэгддэг засварлах програм.
    \item \textbf{Jira / Trello}: Хөгжүүлэлтийн ажлын төлөвлөлт, хяналт, төслийн явцыг хянахад ашиглах менежментийн хэрэгсэл.
    \item \textbf{PyCharm}: Python хөгжүүлэлтэд зориулсан мэргэжлийн IDE.
\end{itemize}
