\chapter{Системийн судалгаа}
\label{chapter1}
\section{Системийн тухай танилцуулга}
%Сонгож авсан системтэй холбоотой үйл ажиллагааны талаарх ерөнхий танилцуулгыг бичнэ. Яг өөрийн хөгжүүлэх системийн ажиллагааны тухай биш. Жишээлбэл, хувьцааны апп хөгжүүлэх бол энэ хэсэгт хувьцаа гэж юу болох, хувьцааг хэрхэн арилжаалдаг гэх мэт ерөнхий суурь ойлголтуудыг тайлбарлана. Харин яг өөрийнхөө хөгжүүлэх апп-тай холбоотой үйл ажиллагааг бүлэг 3-т бичнэ.

%Ашигласан материалыг references файлд хөтөлж явах ба ашиглаж буй хэсэгт cite команд ашиглан оруулна. Жишээ нь:
\par Манай хувцас санал болгох систем нь хиймэл оюун ухаан ашиглан  хэрэглэгчид өдөр тутмын хувцаслалтыг санал болгодог шинэлэг, хэрэглэгчид ээлтэй систем юм. Хэрэглэгчийн оруулсан мэдээлэл болоод олон зуун зурган өгөгдлүүд дээр хиймэл оюун ухаан ашиглан боловсруулалт хийн тухайн өдрийн цаг агаарт зориулан гаргасан хувцаслалтыг хэрэглэгчид хүргэнэ. Мөн хэрэглэгчийн хувцастай зохицох хувцаснуудыг хэрэглэгчид санал болгох болно.

\section{Ижил төстэй системийн судалгаа}
\subsection{Монгол дахь ижил төстэй системүүд}
% Сонгож авсан сэдвийн хүрээнд дотоодод ажилладаг веб, апп байвал судалж, ерөнхий ажиллагаа, системийн үндсэн функц зэргийн тухай бичнэ. Мөн ажиллагааны ерөнхий агшныг харуулсан зураг оруулж болно. Яг сонгож авсан чиглэлд ажиллаж байгаа систем байхгүй бол төстэй систем судалж болно. Дор хаяж хоёр систем судалсан байх ёстой. 
\subsection{Гадаадын ижил төстэй системүүд}
\subsubsection*{ Stitch Fix}
\par Тайлбар: Stitch Fix нь хэрэглэгчийн сонголтод тулгуурлан хувийн загварын зөвлөмж өгдөг онлайн хувийн хэв маягийн үйлчилгээ юм. Үйлчлүүлэгчид өөрсдийн хэв маяг, хэмжээ, загварын сонголтын талаар дэлгэрэнгүй асуулга бөглөдөг. Энэхүү мэдээлэлд үндэслэн хүний стилистууд болон алгоритмуудын хослол нь хувцасны сонгомол сонголтыг санал болгож байна.

\parАрга барил:

\parГибрид систем: Stitch Fix нь хамтын шүүлтүүр (ижил төрлийн сонголттой хэрэглэгчдэд тулгуурласан зүйлсийг санал болгох) болон контентод суурилсан шүүлтүүрийг (хэрэглэгчийн өмнө нь таалагдсан эсвэл худалдаж авсан зүйлстэй төстэй зүйлсийг санал болгох) хослолыг ашигладаг\\.
Санал хүсэлтийн гогцоо: Энэ нь үйлчлүүлэгчдээс хүлээн авсан зүйлсийн талаархи санал хүсэлтийг нэгтгэн зөвлөмжөө тасралтгүй сайжруулдаг.\\
Машины сургалт: Тус компани нь тухайн хэрэглэгчийн түүх болон өгсөн өгөгдөл дээр үндэслэн аль зүйл тохирохыг урьдчилан таамаглах алгоритмыг ашигладаг.\\
Давуу тал: Системийн хүний ойлголтыг машин сургалттай хослуулсан нь хэв маягийн сонголт, сэтгэл санааны байдал зэрэг субъектив хүчин зүйлсийг багтаасан өндөр хувийн туршлагыг бий болгодог.
\subsubsection{Amazon Fashion}
\parТайлбар: Amazon Fashion нь хэрэглэгчдэд өөрсдийн хэв маяг, хэмжээ, сонголтод тохирсон хувцсыг олоход нь туслах зорилгоор төрөл бүрийн зөвлөмж аргачлалуудыг ашигладаг. Зөвлөмжүүд нь Amazon-ийн өргөн хүрээний платформд нэгтгэгдсэн бөгөөд сайт дээрх хэрэглэгчийн зан төлөвт үндэслэсэн болно.\\

Арга барил:

\parХамтарсан шүүлтүүр: Амазон нь ижил төстэй зан үйлтэй хэрэглэгчдийн худалдан авалт эсвэл үзэл бодолд үндэслэн хувцасны зүйлсийг санал болгодог.\\
Гүнзгий суралцах ба хиймэл оюун ухаан: Амазон нь хэрэглэгчийн сонголтод (худалдан авалтын түүх, хайлтын зуршил) дүн шинжилгээ хийдэг гүнзгий сургалтын загваруудыг ашигладаг бөгөөд үүнийг илүү өргөн хүрээний чиг хандлага, түгээмэл байдлын өгөгдөлтэй хослуулдаг.\\
Visual Search: Amazon-ийн "StyleSnap" нь хэрэглэгчдэд зураг байршуулах боломжийг олгодог бөгөөд систем нь платформ дээр худалдаж авах боломжтой ижил төстэй хувцаснуудыг санал болгохын тулд зураг таних аргыг ашигладаг.\\
Давуу тал: Амазоны систем нь асар их хэмжээний хэрэглэгчийн өгөгдөл, гүйлгээний түүхийг ашиглаж, дүрс таних зэрэг дэвшилтэт хиймэл оюун ухааны аргуудыг нэгтгэсэн, өргөтгөх боломжтой.

% Гадаадад ижил төстэй систем олдох тул түүнтэй холбоотой судалсан мэдээллүүдийг оруулна. Дор хаяж хоёр систем судалсан байна.

%Судалсан системүүдийн талаар мэдээллийг нэгтгэсний дараа давуу ба сул талуудыг хүснэгтэд цэгцэлж харьцуулна. 

%-------------------------------------------
\section{Холбогдох хууль, дүрэм журам}
Энэхүү системийг хөгжүүлэхэд мөрдөх шаардлагатай хууль, дүрэм журмуудыг тодорхойлох нь чухал юм. Үүнд дараах хууль, дүрэм, журмууд багтана:
\begin{itemize}
    \item \textbf{Хувийн мэдээллийг хамгаалах тухай хууль}: Хэрэглэгчийн хувийн мэдээллийг хамгаалах, зөвшөөрөлгүй ашиглалт, задруулалт, алдагдлаас сэргийлэх.
    \item \textbf{Цахим худалдааны хууль}: Систем дээр онлайн худалдаа, төлбөрийн үйлчилгээнд хамаарах хууль тогтоомжийг дагаж мөрдөх.
    \item \textbf{Интернэт аюулгүй байдлын дүрэм}: Хэрэглэгчийн өгөгдлийн аюулгүй байдлыг хангах, системийн найдвартай ажиллагааг хангах дүрмүүд.
    \item \textbf{Хэрэглэгчийн эрхийг хамгаалах хууль}: Хэрэглэгчид систем ашиглах үед үүсэх эрх, үүргийг зохицуулах, шударга үйлчилгээ үзүүлэх хууль тогтоомж.
\end{itemize}

%-------------------------------------------
\section{Шаардлага гаргаж авах техник}
Системийн шаардлагыг бүрэн, оновчтой тодорхойлохын тулд шаардлага гаргаж авах олон төрлийн аргуудыг ашиглана. Тодорхойлох аргаас хамааран тухайн оролцогчдын хүсэл, хэрэгцээ, санал бодлыг нарийвчлан гаргаж авах боломжтой.

\subsection{Хэрэглэгч, оролцогчдоос авсан бодит санал асуулга}
Системд тавигдах шаардлагыг тодорхойлоход санал асуулгын аргачлалыг ашигласан. Үүнд:
\begin{itemize}
    \item Хэрэглэгчдийн шаардлагыг ойлгох, хэрэгцээг тодорхойлох зорилготой анхан шатны асуулгын хуудас.
    \item Санал асуулгад үндэслэн системд хэрэгцээтэй үйлдлүүд болон онцлог шинж чанаруудыг гаргаж авах.
\end{itemize}

\subsection{Ярилцлага}
Ярилцлагын аргыг ашиглан системийн оролцогчдоос шаардлагатай мэдээллийг гаргаж авсан. Энэ нь санал асуулгын дүнг дэлгэрэнгүй тодорхойлох, оролцогчдын санаа бодлыг гүнзгий ойлгоход тусалсан. Ярилцлагын аргын ашиг тус:
\begin{itemize}
    \item Оролцогчдын санал бодлыг гүнзгийрүүлэн авч үзэх боломжтой.
    \item Бодит нөхцөл байдал, тулгамдаж буй асуудлыг тодорхойлж, шийдэлд хүрэхэд дөхөмтэй.
    \item Системийн шаардлагыг илүү оновчтой, бүрэн дүүрэн тодорхойлох боломжийг олгоно.
\end{itemize}



